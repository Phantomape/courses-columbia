 \documentclass[a4paper, 11pt]{article}
\usepackage{comment} % enables the use of multi-line comments (\ifx \fi)
\usepackage{lipsum} %This package just generates Lorem Ipsum filler text.
\usepackage{fullpage} % changes the margin
\usepackage{indentfirst}

\begin{document}
%Header-Make sure you update this information!!!!
\noindent
\large\textbf{Report: Computer Graphics} \hfill \textbf{Xucheng Chen} \\
\normalsize COMS 4160 \hfill xc2360 \\
Prof. Zheng \hfill Date: 02/23/2017

\section*{Running the code}
    \indent I basically finished all coding in eclipse, so the easiest way to compile and run the code would be just import the code in eclipse and then configure the build path to include the lwjgl library. After importing the project, just run Main.java:). If you have any  problems running the sample code, please contact me and I will personally demo it in office hours.
%Put your Problem statement here! Example of a Citation\cite[p.219]{Robotics}. Here's Another Citation\cite{Flueck}

\section*{External resources}
    \indent There are some external code I copied from the official tutorial of lwjgl, namely Texture.java, Scene.java, etc. In other words any class that is not included in the initial starter code is from external source and they won't be used in this assignment. These external class are just for test purpose(Unfortunately, they won't work, lol)

\section*{Creative scene}
    \indent The scene is an animation of the space battle in Star Wars. There are four objects in the scene: an Arc170, two TIE-fighters and the Moon. The Moon would be constantly spinning and the two TIE-fighters would be chasing after the Arc170. The highlight of this animation would be the chasing process of these three fighters, because their movement is really hard to generate and sometimes the movement may seem too abrupt. I have tried my best to make this video as authentic as it can be.

\section*{Installiation}
   \indent The following installation guide is copied from the official website of lwjgl: Eclipse supports Gradle/Maven projects and it is highly recommended to use them instead of a manual Eclipse project configuration. However, if you prefer setting up a native Eclipse project, follow these instructions (works with Eclipse Neon):
 
\begin{itemize}
   \item     Download the ZIP bundle from https://www.lwjgl.org/download, which is already included in lib folder
   \item     When the download is complete, extract its contents to some file system directory, henceforth referred to as <lwjgl3>.
   \item     In Eclipse go to menu "Window" > "Preferences" and in the tree view to the left search for 'Java' > 'Build Path' > 'User Libraries' and Click "New..." in the 'User Libraries' dialog. In the opened modal dialog "New User Library" write "LWJGL3" in the 'User library name:' text field and click 'OK'. The newly created library "LWJGL3" should show up in the list 'Defined user libraries:'.
   \item     Now, select this item in the list and click 'Add External JARs...'. This opens a standard OS file selection dialog allowing you to select *.jar files which get added to the classpath/buildpath of all projects using the LWJGL3 User Library. Go to <lwjgl3> and select all *.jar files which do not have -javadoc or -sources in their names. Make sure you don't forget the lwjgl-natives-<os>.jar file, and click 'Open'. This will populate the LWJGL3 user library in the list with respective entries for all selected jar files. You could leave it at that now in order to use LWJGL 3.
   \item     However, if you want to have Sources and JavaDocs, you will have to select each of the entries, click on 'Source attachment: (None)' and on 'Edit...'. This will open the "Source Attachment Configuration" dialog. Here you could select 'External location' and 'External File...' to select the appropriate *-sources.jar file.
   \item     In order to actually use the LWJGL3 User Library in one of your projects, go to the Build Path settings of your project and select the 'Libraries' tab. Here, click 'Add Library...', select 'User Library' and mark the "LWJGL3" User Library.
   \item     Now you are set up to use LWJGL 3 in your project.
\end{itemize}

\section*{Additional files}
    \indent The video is include in the video file.
% to comment sections out, use the command \ifx and \fi. Use this technique when writing your pre lab. For example, to comment something out I would do:
%  \ifx
%   \begin{itemize}
%       \item item1
%       \item item2
%   \end{itemize}
%  \fi

\ifx

\section*{Analysis \& Testing}
\lipsum[6]

\section*{Final Evaluation}
\lipsum[7]

\section*{Attachments}
%Make sure to change these
Lab Notes, HelloWorld.ic, FooBar.ic
%\fi %comment me out

\begin{thebibliography}{9}
\bibitem{Robotics} Fred G. Martin \emph{Robotics Explorations: A Hands-On Introduction to Engineering}. New Jersey: Prentice Hall.
\bibitem{Flueck}  Flueck, Alexander J. 2005. \emph{ECE 100}[online]. Chicago: Illinois Institute of Technology, Electrical and Computer Engineering Department, 2005 [cited 30
August 2005]. Available from World Wide Web: (http://www.ece.iit.edu/~flueck/ece100).
\end{thebibliography}
\fi
\end{document} 